\section{Public Key Infrastructure}%
\label{sec:public_key_infrastructure}

Wofür brauchen wir sowas?
Um Schlüssel zu signieren (Zertifikate erstellen), zu verteilen und falls nötig
zurückzuziehen.
Dies ist nötig um Man-in-the-middle Angriffe zu vereiteln.

Eine PKI erzeugt, verteilt und erklärt Zertifikate für ungültig.
Sie unterstützt die sichere Kommunikation und die Handhabung von rechtlich bindenen
Dokumenten (Signaturen, Unleugbarkeit).
Es liefert Dienste für sichere Zeitstempel.
Weiterhin kann es beim Privilegienmanagement helfen bzgl. der Authorisierung (nicht
Authentifizierung!).
Sie kann auch Teil von einem Digital Rights Management Systems sein.

Das Wesen von Public Key Krypto ist, dass ein Schlüssel zum Verschlüsseln und ein anderer
Schlüssel zum Entschlüsseln benutzt wird.
Der öffentliche Schlüssel muss dabei authentifiziert werden, damit nachgewiesen ist, dass
auch wirklich die Person dahintersteckt, die sie vorgibt zu sein.
Eine Certificate Authority (CA) übernimmt dabei diesen Signierungsprozess.

\subsection{Certificate Revocation}%
\label{sub:certificate_revocation}

Was passiert, wenn private Schlüssel verloren gehen oder gestohlen werden?
Verlorene private Schlüssel können nicht wiederhergestellt werden.
Gestohlene private Schlüssel sind ein Sicherheitsrisiko.

Also müssen neue Schlüssel erzeugt und die alten widerrufen werden (revoke).
Die Kommunikationspartner müssen dabei informiert werden, dass die widerrufenen Schlüssel
nicht mehr benutzt werden.
Dafür bietet eine PKI eine Liste von widerrufenen Schlüsseln: die Certificate Revocation
List (CRL).
Beispiel: \url{http://crl.verisign.com/Class3InternationalServer.crl}

\subsection{Certificate Distribution}%
\label{sub:certificate_distribution}

Wie kommen die Kommunikationspartner an die jeweiligen Schlüssel des anderen?
Sie können sich z.\,B. die Schlüssel einfach gegenseitig schicken.
Das funktioniert, weil die CA die Gültigkeit der Schlüssel gewährleistet.
Die Schlüssel können aber auch über eine Certificate Management Facility verwaltet werden,
sodass sich die Partner die Schlüssel problemlos herunterladen können.

Dafür wird normalerweise X.509 und LDAP genutzt.
X.509 wird dabei genutzt, um die Interoperabilität zwischen verschiedenen Anwendungen zu
gewährleisten (Webserver, E-Mail, VPN-Gateways).
Ein X.509 Zertifikat enthält dabei bspw. folgende Informationen:
\begin{itemize}
  \item Version
  \item Seriennummer
  \item Signatur
  \item ausstellende CA
  \item ID der ausstellenden CA
  \item Gültigkeitsdauer
  \item Inhaber
  \item Schlüsselinformationen des Inhabers: Public-Key-Algorithmus, Public-Key des
    Inhabers
  \item genutzter Signaturalgorithmus
  \item Signatur
\end{itemize}

Andere Formate: Public Key Cryptography Standards (PKCS).
