\section{Hardware Security Modules}%
\label{sec:hardware_security_modules}

Für eine CA müssen die privaten Schlüssel sicher gelagert werden.
Nur die richtigen Personen dürfen auf die Schlüssel zugreifen.
Die Schlüssel dürfen nicht verändert werden und müssen gegen Diebstahl und Verlust
geschützt werden.
Dabei reicht es nicht aus, die Schlüssel als normale Datei auf einer Festplatte zu
speichern, da dies bestimmte Angriffe erlaubt.

\subsection{Speicherung eines Schlüssels}%
\label{sub:speicherung_eines_schlussels}

Schlüssel können entweder eingegeben werden, wenn das Gerät gestartet wurde, oder sie
können in einem Festspeicher (HDDs, SSDs, ROMs, Silikonstrukturen, RAM) gespeichert
werden.
Wenn sie aber irgendwo abgespeichert werden, sind sie anfällig für Lauschangriffe
(\emph{eavesdropping)}.

Dafür gibt’s jetzt spezielle Module: Hardware Security Module.
Ein HSM ist ein physikalisches Rechengerät (physical computing device),
das digitale Schlüssel verwaltet und schützt,
Verschlüsselungs- und Entschlüsselungsfunktionen für digitale Signaturen,
starke Authentifizierung und andere kryptographische Funktionen ausführt.
Das sind normalerweise Einsteckkarten oder externe Geräte, die man an einen Rechner oder
einen Netzwerkserver anschließt.
Ein HSM enthält einen oder mehrere sichere Kryptoprozessorchips.
Man kann einen Schlüssel auf dem HSM erzeugen, aber es gibt keine Funktionen, um die
Schlüssel zu extrahieren.
Einfaches Beispiel: eine Kreditkarte mit NFC.

\subsection{Funktionen eines HSM}%
\label{sub:funktionen_eines_hsm}

Sie erzeugen onboard kryptographische Schlüssel.
Dafür sind viele Zufallszahlen nötig und richtige Zufallszahlen sind auf einem
deterministischem Computer schwierig zu erzeugen.
Sie speichern die Schlüssel sicher ab – zumindest die sensitivsten Schlüssel (master
keys).
Sie führen die Verschlüsselung oder digitale Signierung aus.

Wichtig ist dabei: achte auf Seitenkanalangriffe!

\subsection{Backup eines Schlüssels}%
\label{sub:backup_eines_schlussels}

Dabei gibt es zwei Methoden:
die \emph{multiple custodians method}
und die \emph{single custodian method}.

\paragraph{multiple custodians}%
\label{par:multiple_custodians}

Der Schlüssel wird in Teile gespalten und auf mehrere Verwahrer (custodians) verteilt.
Die Teile werden von einem zweiten zufällig gewählten Schlüssel (wrapping key)
verschlüsselt (verpackt).
Zur Wiederherstellung werden entweder beim Standardverfahren alle Verwahrer
oder bei einem $N$ aus $M$ Schema ein bestimmtes Minimum der Verwahrer benötigt.

\paragraph{single custodian}%
\label{par:single_custodian}

Hierbei wird der Schlüssel von einem speziell ausgewählten Schlüssel (wrapping key)
verschlüsselt.
