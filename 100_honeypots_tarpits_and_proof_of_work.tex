\section{Honeypots, Tarpits und Proof of Work}%
\label{sec:honeypots_tarpits_und_proof_of_work}

\subsection{Honeypots}%
\label{sub:honeypots}

\begin{definition}
  Ein \emph{Honeypot} ist ein Sicherheitsmechanismus, der unauthorisierte Nutzungsversuche
  von Informationssystemen erkennen, ablenken und in gewisser Weise entgegenwirken soll.
\end{definition}

Im Unterschied zu einem IDS ist ein Honeypot aktiv, während ein IDS passiv ist.

Es gibt Honeypots aus Forschungszwecken bspw. für Anti-Viren-Software, der über neue Viren
und Varianten lernen will.
Netzwerkadmins wollen Informationen über aktuelle Bedrohungen sammeln und nutzen
normalerweise ein großes verteiltes Netzwerk von Fallen (auch als \emph{Honeynet}
bekannt).
Im produktiven Einsatz sollen Honeypots die Aufmerksamkeit von gut bekannten Angriffen auf
sich lenken und vom eigentlich Ziel ablenken.
Das ist nicht wirklich eine akzeptable Methode aus der Sicherheitsperspektive, aber es
hilft.

\paragraph{Wirkprinzipien}%
\label{par:wirkprinzipien}

Honeypots verhalten sich wie ein richtiges System.
Einem Angreifer sollte nicht auffallen, dass er mit einem Honeypot interagiert.
Sie zeichnen alle Nutzeraktionen auf, sodass Admins diese Spuren genau verfolgen können.
Danach analyieren Admins diese Angriffsmuster.

Honeynets beobachten nur die Netzwerkaktivitäten.

Spam traps sind veröffentlichte E-Mail-Adressen, die nur Spam anziehen sollen.
Die Trainingsdaten können später zur Spamerkennung genutzt werden.

Virus traps beobachten alle Virenaktivitäten in einer sand box.

Ein tarpit ist wie ein Honeypot, nur dass aktive Funktionen besitzt.
Er verlangsamt Angreifer, indem künstliche Wartezeiten in Protokollen eingebaut werden.
Damit werden die Kosten von Angriffen erhöht.

\paragraph{Proof of Work}%
\label{par:proof_of_work}

Ein POW-System fügt jedem Dienst einen gewissen Preis hinzu.
Dies kann bspw. CPU-Zeit für die Lösung eines Problems sein, welches sogar praktischen
Nutzen haben könnte.
Die Lösung des Problem muss aber schnell überprüfbar sein.

Beispiel: Wurzel modulo $p$.
Wähle eine Primzahl $p$ mit vernünftiger Größe (bspw. $1024$ bits).
Die Aufgabe besteht daraus zu einem gegebenen $x$ eine Zahl $y$ zu finden mit
$y^2 \equiv x \mod p$.
Sei bspw. $p=97$ und $x=50$, dann ist $y=27$.

Das kostet natürlich Energie, aber dämmt Spam ein.
Lohnt sich das?
Energie ist dabei die Währung.

Mit Application Specific Integrated Circuits (ASICs) können solche schwierigen Probleme
schneller gelöst werden.
Das führt zu einem Katz-und-Maus-Spiel.

Diese Arbeit kann für andere Angriffe benutzt werden (bspw. können CAPTCHAs ausgenutzt
werden).
