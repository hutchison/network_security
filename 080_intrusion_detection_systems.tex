\section{Intrusion Detection Systems}%
\label{sec:intrusion_detection_systems}

IDS sind eine weitere Möglichkeiten um herauszufinden, was im Netzwerk abläuft.
Ein IDS verhindert keine Angriffe, aber erkennt Indizien für welche.

Ein IDS \emph{schüffelt} (sniffs) den Traffic, analysiert ihn und sucht nach Spuren für
\begin{itemize}
  \item Scans
  \item Aufklärungsaktivitäten
  \item Exploits
\end{itemize}
Dabei ist ein IDS voll transparent.

\subsection{Motivation}%
\label{sub:motivation}

Ohne ein IDS würden Admins Angriffe überhaupt nicht bemerken.
Folgende Infos werden gebraucht:
\begin{itemize}
  \item Welche Güter (assets) wurden angegriffen?
    Welcher Host wurde angreiffen?
    Server, Medien, User devices, Personen?
  \item Welche Daten wurden kompromittiert?
    Persönliche, medizinische, Logindaten, Zahlungen?
  \item Welche Methode wurde zum Angriff genutzt?
    Welcher Angriffsvektor wurde gewählt?
    Webseite, physisch, Backdoor?
\end{itemize}
Angriffe brauchen meist mehrere Einzelschritte, daher könnten Folgeschritte vermieden
werden.

\subsection{Anwendungen}%
\label{sub:anwendungen}

\begin{itemize}
  \item Betrugserkennung für Kreditkarten anhand der Zahlungsgeschichte.
  \item Einbruchserkennung durch Erkennung unnormalen Verhaltens.
\end{itemize}

\subsection{Typen}%
\label{sub:typen}

\paragraph{Netzwerkbasiert}%
\label{par:netzwerkbasiert}

Die Sensoren eines NIDS werden an bestimmten Punkten im Netzwerk (z.\,B. hinter der
Firewall) platziert und beobachten den Traffic von und zu allen Geräten.

\paragraph{Hostbasiert}%
\label{par:hostbasiert}

Ein HIDS läuft auf einzelnen Hosts/Geräten im Netzwerk und beobachten nur den Traffic des
Hosts.

\subsection{Methoden}%
\label{sub:methoden}

Um unnormalen Traffic zu erkennen werden z.\,B. \emph{statistische} Methoden benutzt.
Als Parameter nimmt man bspw: Ziel, Datenrate, Ports und Zeit.
Dann benutzt man Bayesianische Filter um die Wahrscheinlichkeit (likelihood) von Traffic
zu berechnen unnormal zu sein.
Diese Filter werden an normalem Traffic trainiert.

Eine andere Möglichkeit ist die Signaturerkennung.
Dabei werden Pakete nach bestimmten Mustern durchsucht.
Ein Beispiel: ein HTTP Request enthält
\begin{verbatim}
  get /scripts/..%c0%af../winnt/system32/cmd.exe?c+dir
\end{verbatim}
Dies wurde vom Nimda-Wurm benutzt um bestimmte Versionen von MS IIS zu exploiten.

Bei diesen Methoden kommt es immer zu falsch positiven und falsch negativen Warnung.
Wenn dies zu oft vorkommt, entsteht eine gewisse Ignoranz bei den Administratoren.
Daher müssen solche Regeln stetig verfeinert werden.
