\section{Proxy Firewalls}%
\label{sec:proxy_firewalls}

Bisher können wir nur einzelne Pakete filtern.
Wie können aber noch nicht die Inhalte der Verbindung filtern (z.\,B. Downloads).
Dafür braucht man Proxies.

Ein Proxy agiert für den internen Client und den externen Server.
Auf ihm laufen nur ein paar sichere und vertrauenswürdige Programme.
Ein Proxy kann die Inhalte der Verbindung bzw. der Pakete verändern, was zu ethischen
Problemen führen kann.

\subsection{Arten von Proxies}%
\label{sub:arten_von_proxies}

\paragraph{Forward Proxy}%
\label{par:forward_proxy}

Der Client befindet sich im internen Netzwerk und startet die Verbindung zum externen
Server.
Der Proxy nimmt die Verbindung entgegen und startet eine eigene Verbindung zum Server.

\paragraph{Reverse Proxy}%
\label{par:reverse_proxy}

Der Client befindet sich im externen Netzwerk und startet die Verbindung zum internen
Server.
Der Proxy nimmt die Verbindung entgegen und startet eine eigene Verbindung zum Server.
Wird benutzt um Dienste abzusichern, da jeder eingehende Traffic vom Proxy überwacht wird.

\subsection{Funktionsweise}%

Siehe \hyperref[sub:arten_von_proxies]{Arten von Proxies}.
Client und Server agieren niemals direkt miteinander.

Proxies können transparent oder intransparent sein.
Hier sind sie immer intransparent.
Die Clients müssen daher für Proxies konfiguriert werden, damit alles funktioniert.

Bei \emph{application level proxies} wird für jeden Dienst ein Proxy implementiert, sodass
nur bestimmter Traffic erlaubt ist, wie z.\,B.:
\begin{itemize}
  \item bestimmte Nutzer (wäre mit transparenten Firewalls unmöglich)
  \item bestimmte Adressen
  \item bestimmte Computer
\end{itemize}

Beispiel: \href{https://de.wikipedia.org/wiki/SOCKS}{SOCKS}

Da ein Proxy von Außen erreichbar ist, muss er gegen Angriffe abgehärtet sein.
Proxies sind normalerweise dual-homed (haben zwei Netzwerkschnittstellen).
Die interne Struktur des Netzwerks bleibt verborgen.
Auch passives Fingerprinting ist unmöglich (OS-Erkennung durch die Standardeinstellungen
bei Paketen wie z.\,B. \href{https://de.wikipedia.org/wiki/Time_to_Live}{TTL},
\href{https://de.wikipedia.org/wiki/TCP_Receive_Window}{TCP Receive Window Size},
TCP Optionen).
Ein \emph{bastion host} liefert nur die nötigsten Infos und bietet höchste Sicherheit.

\paragraph{Probleme}

Proxies haben Probleme mit verschlüsselten Verbindungen.
Wenn der Proxy Zugriff auf den Inhalt haben will, muss er diesen entschlüsseln, was aber
dank Ende-zu-Ende-Verschlüsselung unmöglich ist.

\paragraph{Vor- und Nachteile}%
\label{par:vor_und_nachteile}

Vorteile:
\begin{itemize}
  \item interne Struktur bleibt vor der externen Welt verborgen
  \item Traffic kann einfach überwacht werden
  \item nutzerbasierte Sicherheit ist möglich, Authentifizierung kann implementiert werden
  \item Spoofing wird verhindert, da der Proxy den ganzen Traffic der Außenwelt erzeugt
\end{itemize}

Nachteile:
\begin{itemize}
  \item Performance reduction
  \item Proxy ist ein single point of failure/attack
  \item application proxies müssen für jede Application entwickelt werden
  \item Software muss angepasst werden
  \item bastion host muss gehärtet werden
\end{itemize}
