\section{Policies}%
\label{sec:policies}

Was ist eine Policy?
Eine Sicherheitspolicy definiert, was getan werden muss um auf Computern gespeicherte
Informationen zu schützen.
Sie definiert \emph{was} zu tun ist und \emph{wie} es evaluiert werden kann.
Wenn man nicht wahrnimmt, dass Policies missachtet werden, dann sind sie sinnlos.
Daher müssen sie evaluiert werden.
Eine Policy wird normalerweise aufgeschrieben.

Die Regeln einer Firewall repräsentieren eine Policy, sodass
\begin{itemize}
  \item alles verboten ist, außer es ist explizit erlaubt;
  \item alles erlaubt ist, außer es ist explizit verboten.
\end{itemize}
Die Wahrheit ist: alles ist verboten,
außer es ist explizit erlaubt oder es kommt trotzdem durch.

\subsection{Komplexität von Policies}%
\label{sub:komplexitat_von_policies}

Policies werden von Menschen definiert und befolgt.
Daher müssen sie verständlich sein.
Wenn sie zu komplex sind, dann sind sie nicht durchsetzbar.

Ein anderes Extrem ist jedoch: „Firmencomputer dürfen nicht für den persönlichen Gebrauch
genutzt werden!“
Diese Regel ist zu simpel und zu uneindeutig und daher nicht durchsetzbar.

Besser ist:
\begin{itemize}
  \item Sende keine Kettenbriefe.
  \item Sende keine Dateien.
  \item Nutze keine Pornoseiten.
\end{itemize}

\subsection{Entwicklung von Policies}%
\label{sub:entwicklung_von_policies}

Best practice:
\begin{enumerate}
  \item Risiken identifizieren
  \item Kommuniziere Funde
  \item erstelle oder aktualisiere Policies
  \item bestimme die Zustimmung der Policy mit den Mitarbeitern
  \item versuche eine \emph{Kultur} zu erschaffen
\end{enumerate}

\paragraph{Risiken identifizieren}%

Mach eine Sicherheitsanalyse: bestimme kritische Daten und Systeme und analysiere die
normale Nutzung des Netzwerks.
Schreib es auf.

\paragraph{Kommuniziere Funde}%

Berichte an’s Management: einfach, balanciert, kurz, zeige nicht auf einzelne Personen,
halte dich allgemein.

\paragraph{Erstelle oder aktualisiere Policies}%

Schreib sie auf und zwar spezifisch und klar.
Was muss gemacht werden?
Warum?
Wer ist verantwortlich?

Niemand liest mehr als 10 Seiten von Policies.
SMART: specific, measurable, achievable, realistic, time-based.

\paragraph{Bestimme die Zustimmung der Policy}%

Wenn eine Regel nicht messbar ist, dann ist sie nicht durchsetzbar.
Mögliche Inspektionen (audits): Stichproben, Analyse der Logfiles, Festplatten
durchsuchen.
Aber behalte den menschlichen Aspekt im Auge.
Niemand möchte gerne seine Browserhistory preisgeben.

\paragraph{Kulturelle Aspekte}%

Rede mit den Nutzern über die Risiken.
Erkläre die Policy bevor sie durchgesetzt wird.
Versuche „Anordnungen“ und authoritäres Verhalten zu vermeiden.
