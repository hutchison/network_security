\section{Enterprise Authentication}%
\label{sec:enterprise_authentication}

\begin{definition}
  \emph{Authentifizierung} ist der Prozess die Identität oder eine Eigenschaft (claim)
  von einer Entität (bspw. einer Person) nachzuweisen.
\end{definition}

\subsection{Prinzipien}%
\label{sub:prinzipien}

\begin{description}
  \item[wissensbasiert] PIN, Passwort, persönliche Informationen
  \item[besitzbasiert] SmartCard / SIM Karte, Kreditkarte, Schlüssel, RFID Token, Telefon,
    PC, DRM Modul
  \item[biometrisch] Retina, Iris, Fingerabdruck, DNA, Stimme, Unterschrift
\end{description}
Die Sicherheit wird durch Zwei-Faktor-Authentifizierung weiter verbessert, indem zwei
Methoden kombiniert werden.

Probleme der einzelnen Prinzipien:
\begin{description}
  \item[wissensbasiert] die Nutzer wählen ein schwaches Passwort,
    Nutzer vergessen die Passwörter, Passwort wurde anderweitig kompromittiert
  \item[besitzbasiert] Nutzer verlieren Token, Token wird kopiert
  \item[biometrisch] bestimmte biometrische Eigenschaften können kopiert werden,
    hat ethische Probleme (DNA liefert Aufschlüsse über Gesundheitszustand),
    schwierig zu handhaben
\end{description}

\subsection{Implementierungen}%
\label{sub:implementierungen}

\subsubsection{Kerberos}

Hier wird Authentifizierung und Authorisierung und der eigentliche Dienst voneinander
getrennt.


\subsubsection{RADIUS}
