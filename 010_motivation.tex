\section{Motivation}%
\label{sec:motivation}

\begin{definition}[Netzwerk]
  Ein \emph{Netzwerk} ist eine Gruppe von Computern, die bekannte Kommunikationsprotokolle
  über digitale Verbindungen nutzen um Ressourcen zu teilen, die von den Netzwerkknoten
  bereitgestellt werden.

  Die Verbindungen zwischen den Knoten basieren auf physisch verkabelten, optischen, und
  kabellosen Methoden.
\end{definition}

Computer sind angreifbar; ein vernetztes System ist um ein Vielfaches angreifbarer.
Es ist unmöglich alle Computer im Internet abzusichern, aber es ist sinnvoll und machbar
die eigenen Computer abzusichern.

\begin{definition}[Intranet]
  Ein \emph{Intranet} ist ein eine Menge von Netzwerken, die von einer einzelnen
  administrativen Einheit kontrolliert wird.
\end{definition}

\subsection{Typen, Motivationen, Fähigkeiten von Angreifern}%
\label{sub:typen_motivationen_fahigkeiten_von_angreifern}

Einzelne Angreifer
\begin{itemize}
  \item mit einem bestimmten sozialen Hintergrund,
  \item werden angetrieben durch Publicity,
  \item möchten eventuell politische Statements setzen,
  \item nehmen nur geringe Risiken auf.
\end{itemize}
Organisierte Verbrecher
\begin{itemize}
  \item werden angetrieben durch Geld,
  \item nehmen mittlere Risiken auf.
\end{itemize}
Terroristen
\begin{itemize}
  \item sind politisch und sozial motiviert,
  \item wollen kein Geld,
  \item nehmen hohe Risiken auf (bis zur Lebensgefahr),
  \item wollen zerstören oder verwirren.
\end{itemize}
Konkurrenten
\begin{itemize}
  \item wollen Informationen stehlen oder zerstören,
  \item bemessen die aufgenommene Risiken nach dem Wert der Informationen.
\end{itemize}
Regierungsorganisationen
\begin{itemize}
  \item betreiben Industriespionage für heimische Unternehmen,
  \item betreiben Militärspionage und hybride Kriegsführung
\end{itemize}

\subsection{Angriffe auf Computer}%
\label{sub:angriffe_auf_computer}

\textbf{Stehlen von Informationen}, aufgrund von
\begin{itemize}
  \item Wettbewerbsvorteile erlangen,
  \item Peinlichkeiten erzeugen,
  \item Erpressung
\end{itemize}
\textbf{Zerstörungen} für
\begin{itemize}
  \item Spaß und Selbstverherrlichung
  \item politische Botschaften
\end{itemize}
\textbf{Sammeln von Informationen}, die an den Angreifer gesendet werden.
Angreifer brauchen Zugang durch
\begin{itemize}
  \item social engineering
  \item Viren, Trojaner, Würmer
  \item physisches Stehlen von Speichermedien
  \item sniffing
\end{itemize}

\textbf{Viren} infizieren Dateien oder auch das ganze System bzw. den boot record.

\textbf{Würmer} verteilen sich durch E-Mails und haben als Payload bspw. einen Virus oder
Trojaner.
Netzwerk-Würmer verteilen sich durch das Ausnutzen von bekannten Softwareschwachstellen
(z.\,B. buffer overflows).
Stadien eines Wurms:
\begin{itemize}
  \item Zielauswahl
  \item exploit
  \item Infektion
  \item Verteilung
\end{itemize}
\textbf{Hintertüren und Trojaner} sind nützliche Software, die bösartige Funktionalitäten
enthalten, z.\,B.:
\begin{itemize}
  \item Logging
  \item Zerstören
  \item Installation von weiterer Software (DoS clients, root kits)
\end{itemize}
\textbf{Manipulation der Identität} (identity spoofing)
\begin{itemize}
  \item Angreifer gibt eine andere Identität vor
  \item normalerweise teilen sich Angreifer und Ziel ein Netzwerksegment
  \item Angreifer verteilt falsche Informationen über Routen und Namen
  \item alle Antworten in einem Protokoll können unter spoofing leiden
\end{itemize}
\textbf{Denial of Service}
\begin{itemize}
  \item Angreifer will einen Dienst überladen und sendet viele Anfragen, die von der
    Gegenseite nicht rechtzeitig genug bearbeitet werden können
  \item \emph{böse} Anfragen können aber nicht von normalen unterschieden werden
  \item Beispiele: HTTP, DNS DoS, SYN Flooding
\end{itemize}
\textbf{Botnetzwerk}
\begin{itemize}
  \item viele Computer werden aus der Ferne kontrolliert
  \item für distributed DoS, Berechnungen für kryptografische Prozesse
    (Schlüssel rückrechnen), Bitcoins minen
\end{itemize}
\textbf{Angriffe auf Passwörter und Schlüssel} durch
\begin{itemize}
  \item brute force
  \item Raten, Wörterbuchattacken
  \item fehlerhafte Implementation, z.\,B. Passwörter im Klartext gespeichert
\end{itemize}
\textbf{Port- bzw. Netzwerkscannen}
\begin{itemize}
  \item findet Löcher im Sicherheitssystem und passende Ziele
  \item sniffing, mapping, port scans
  \item durch Managementprotokolle wie
    \href{https://de.wikipedia.org/wiki/Simple_Network_Management_Protocol}{SNMP} und
    \href{https://de.wikipedia.org/wiki/Internet_Control_Message_Protocol}{ICMP}
\end{itemize}
\textbf{Session highjacking}
\begin{itemize}
  \item Angreifer bricht in eine bestehende Sitzung ein
\end{itemize}
