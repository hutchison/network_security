\section{Stateful Firewalls}%
\label{sec:stateful_firewalls}

\subsection{Funktionsweise}%
\label{sub:funktionsweise}

Stateful Firewalls \emph{kennen} den Zustand der Verbindungen und \emph{wissen} welche
Pakete in diesem Zustand erwartet werden.
Darauf können Regeln definiert werden, die nur in bestimmten Zuständen aktiv werden.
Stateful Firewalls arbeiten hauptsächlich auf dem transport layer (OSI Layer 4), aber auch
auf höheren Layern (OSI Layer $> 4$, wird dann \emph{stateful inspection}) genannt.
Zur Erinnerung: TCP ist Layer 4, HTTP arbeitet auf application layer (7).

\paragraph{Multi-layer inspection}%

Viele Protokolle basieren auf Protokollen einer geringeren Ebene.
Eine stateful Firewall kann mehrere Ebenen überwachen.

\paragraph{Neue Protokolle}%

Neue Protokolle müssen für die Konfiguration der Firewall beachtet werden.
Ein Beispiel sind Multimediaprotokolle, die viele Verbindungen zu externen Computern
herstellen, nachdem die initiale Verbindung hergestellt wurde.

\paragraph{Probleme}%
\label{par:probleme}

Hochperformante Firewalls brauchen manchmal geclusterte Hardware, aber stateful Firewalls
können nicht so einfach geclustert werden, da sie über einen gemeinsamen Zustand verfügen
müssen.

Wie können stateless Protokolle (UDP, DNS, ICMP, HTTP) von stateful Firewalls
gehandhabt werden?
Auch stateless Protokolle definieren, welche Pakete erwartet werden.
Mittels Timeouts können Pseudoverbindungen erzeugt werden.
